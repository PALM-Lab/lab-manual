% Options for packages loaded elsewhere
\PassOptionsToPackage{unicode}{hyperref}
\PassOptionsToPackage{hyphens}{url}
%
\documentclass[
]{book}
\usepackage{amsmath,amssymb}
\usepackage{iftex}
\ifPDFTeX
  \usepackage[T1]{fontenc}
  \usepackage[utf8]{inputenc}
  \usepackage{textcomp} % provide euro and other symbols
\else % if luatex or xetex
  \usepackage{unicode-math} % this also loads fontspec
  \defaultfontfeatures{Scale=MatchLowercase}
  \defaultfontfeatures[\rmfamily]{Ligatures=TeX,Scale=1}
\fi
\usepackage{lmodern}
\ifPDFTeX\else
  % xetex/luatex font selection
\fi
% Use upquote if available, for straight quotes in verbatim environments
\IfFileExists{upquote.sty}{\usepackage{upquote}}{}
\IfFileExists{microtype.sty}{% use microtype if available
  \usepackage[]{microtype}
  \UseMicrotypeSet[protrusion]{basicmath} % disable protrusion for tt fonts
}{}
\makeatletter
\@ifundefined{KOMAClassName}{% if non-KOMA class
  \IfFileExists{parskip.sty}{%
    \usepackage{parskip}
  }{% else
    \setlength{\parindent}{0pt}
    \setlength{\parskip}{6pt plus 2pt minus 1pt}}
}{% if KOMA class
  \KOMAoptions{parskip=half}}
\makeatother
\usepackage{xcolor}
\usepackage{color}
\usepackage{fancyvrb}
\newcommand{\VerbBar}{|}
\newcommand{\VERB}{\Verb[commandchars=\\\{\}]}
\DefineVerbatimEnvironment{Highlighting}{Verbatim}{commandchars=\\\{\}}
% Add ',fontsize=\small' for more characters per line
\usepackage{framed}
\definecolor{shadecolor}{RGB}{248,248,248}
\newenvironment{Shaded}{\begin{snugshade}}{\end{snugshade}}
\newcommand{\AlertTok}[1]{\textcolor[rgb]{0.94,0.16,0.16}{#1}}
\newcommand{\AnnotationTok}[1]{\textcolor[rgb]{0.56,0.35,0.01}{\textbf{\textit{#1}}}}
\newcommand{\AttributeTok}[1]{\textcolor[rgb]{0.13,0.29,0.53}{#1}}
\newcommand{\BaseNTok}[1]{\textcolor[rgb]{0.00,0.00,0.81}{#1}}
\newcommand{\BuiltInTok}[1]{#1}
\newcommand{\CharTok}[1]{\textcolor[rgb]{0.31,0.60,0.02}{#1}}
\newcommand{\CommentTok}[1]{\textcolor[rgb]{0.56,0.35,0.01}{\textit{#1}}}
\newcommand{\CommentVarTok}[1]{\textcolor[rgb]{0.56,0.35,0.01}{\textbf{\textit{#1}}}}
\newcommand{\ConstantTok}[1]{\textcolor[rgb]{0.56,0.35,0.01}{#1}}
\newcommand{\ControlFlowTok}[1]{\textcolor[rgb]{0.13,0.29,0.53}{\textbf{#1}}}
\newcommand{\DataTypeTok}[1]{\textcolor[rgb]{0.13,0.29,0.53}{#1}}
\newcommand{\DecValTok}[1]{\textcolor[rgb]{0.00,0.00,0.81}{#1}}
\newcommand{\DocumentationTok}[1]{\textcolor[rgb]{0.56,0.35,0.01}{\textbf{\textit{#1}}}}
\newcommand{\ErrorTok}[1]{\textcolor[rgb]{0.64,0.00,0.00}{\textbf{#1}}}
\newcommand{\ExtensionTok}[1]{#1}
\newcommand{\FloatTok}[1]{\textcolor[rgb]{0.00,0.00,0.81}{#1}}
\newcommand{\FunctionTok}[1]{\textcolor[rgb]{0.13,0.29,0.53}{\textbf{#1}}}
\newcommand{\ImportTok}[1]{#1}
\newcommand{\InformationTok}[1]{\textcolor[rgb]{0.56,0.35,0.01}{\textbf{\textit{#1}}}}
\newcommand{\KeywordTok}[1]{\textcolor[rgb]{0.13,0.29,0.53}{\textbf{#1}}}
\newcommand{\NormalTok}[1]{#1}
\newcommand{\OperatorTok}[1]{\textcolor[rgb]{0.81,0.36,0.00}{\textbf{#1}}}
\newcommand{\OtherTok}[1]{\textcolor[rgb]{0.56,0.35,0.01}{#1}}
\newcommand{\PreprocessorTok}[1]{\textcolor[rgb]{0.56,0.35,0.01}{\textit{#1}}}
\newcommand{\RegionMarkerTok}[1]{#1}
\newcommand{\SpecialCharTok}[1]{\textcolor[rgb]{0.81,0.36,0.00}{\textbf{#1}}}
\newcommand{\SpecialStringTok}[1]{\textcolor[rgb]{0.31,0.60,0.02}{#1}}
\newcommand{\StringTok}[1]{\textcolor[rgb]{0.31,0.60,0.02}{#1}}
\newcommand{\VariableTok}[1]{\textcolor[rgb]{0.00,0.00,0.00}{#1}}
\newcommand{\VerbatimStringTok}[1]{\textcolor[rgb]{0.31,0.60,0.02}{#1}}
\newcommand{\WarningTok}[1]{\textcolor[rgb]{0.56,0.35,0.01}{\textbf{\textit{#1}}}}
\usepackage{longtable,booktabs,array}
\usepackage{calc} % for calculating minipage widths
% Correct order of tables after \paragraph or \subparagraph
\usepackage{etoolbox}
\makeatletter
\patchcmd\longtable{\par}{\if@noskipsec\mbox{}\fi\par}{}{}
\makeatother
% Allow footnotes in longtable head/foot
\IfFileExists{footnotehyper.sty}{\usepackage{footnotehyper}}{\usepackage{footnote}}
\makesavenoteenv{longtable}
\usepackage{graphicx}
\makeatletter
\def\maxwidth{\ifdim\Gin@nat@width>\linewidth\linewidth\else\Gin@nat@width\fi}
\def\maxheight{\ifdim\Gin@nat@height>\textheight\textheight\else\Gin@nat@height\fi}
\makeatother
% Scale images if necessary, so that they will not overflow the page
% margins by default, and it is still possible to overwrite the defaults
% using explicit options in \includegraphics[width, height, ...]{}
\setkeys{Gin}{width=\maxwidth,height=\maxheight,keepaspectratio}
% Set default figure placement to htbp
\makeatletter
\def\fps@figure{htbp}
\makeatother
\setlength{\emergencystretch}{3em} % prevent overfull lines
\providecommand{\tightlist}{%
  \setlength{\itemsep}{0pt}\setlength{\parskip}{0pt}}
\setcounter{secnumdepth}{5}
\usepackage{booktabs}
\ifLuaTeX
  \usepackage{selnolig}  % disable illegal ligatures
\fi
\usepackage[]{natbib}
\bibliographystyle{plainnat}
\IfFileExists{bookmark.sty}{\usepackage{bookmark}}{\usepackage{hyperref}}
\IfFileExists{xurl.sty}{\usepackage{xurl}}{} % add URL line breaks if available
\urlstyle{same}
\hypersetup{
  pdftitle={Lab Manual for the PALM Lab},
  pdfauthor={Dr William XQ Ngiam},
  hidelinks,
  pdfcreator={LaTeX via pandoc}}

\title{Lab Manual for the PALM Lab}
\author{Dr William XQ Ngiam}
\date{Last updated on 28-12-2023}

\begin{document}
\maketitle

{
\setcounter{tocdepth}{1}
\tableofcontents
}
\hypertarget{welcome}{%
\chapter*{Welcome!}\label{welcome}}
\addcontentsline{toc}{chapter}{Welcome!}

\includegraphics{images/banner_1.png}

This is the lab manual for the \textbf{P}erception, \textbf{A}ttention, \textbf{L}earning and \textbf{M}emory \textbf{Lab} at the University of Adelaide, run by Dr William Ngiam! We are a cognitive psychology and neuroscience lab researching how we keep information in mind for ongoing cognition.

This lab manual contains guidelines, policies, and other resources for how we conduct our research. It is a continual work in progress, as the way we conduct our scientific research evolves.

\begin{center}\rule{0.5\linewidth}{0.5pt}\end{center}

Thank you to the following who inspired this lab manual:

\begin{itemize}
\tightlist
\item
  \href{https://github.com/alylab/labmanual/blob/master/aly-lab-manual.pdf}{Aly Lab}
\item
  \href{https://www.cns.nyu.edu/malab/lablife.html}{Wei Ji Ma Lab}
\item
  \href{https://bookdown.org/jordandanielsnyder/LabManual/expectations-and-responsiblities.html}{Global Community Wellness Lab}
\end{itemize}

\begin{center}\rule{0.5\linewidth}{0.5pt}\end{center}

The PALM Lab Manual written by William Ngiam and members of the PALM Lab is licensed under CC BY-NC 4.0

\hypertarget{the-lab}{%
\chapter{The PALM Lab}\label{the-lab}}

This chapter gives an overview of the PALM Lab - the overarching goal of our lab, our approach to conducting science, and our values.

\hypertarget{lab-mission}{%
\section{Our mission}\label{lab-mission}}

Our attention - broadly, the ability to select which information to focus on and keep in mind - is critical for everyday cognition. But what we can attend to and keep in mind is limited, making what we do retain incredibly valuable. In today's digital age, an endless amount of information is available at our fingertips, with social media and artificial intelligence designing algorithms that disarm and capture our attention in various ways. The overarching goal of our lab is to conduct rigorous science that leads to a \textbf{better understanding of human attention and working memory}, with the hope that it will help improve the collective attention of society to tackle greater problems.

\hypertarget{lab-philosophy}{%
\section{Our philosophy of science}\label{lab-philosophy}}

We believe that science is best when it is conducted with integrity and with the common good in mind. The PALM Lab strives to uphold this belief through a commitment to rigorous and transparent scholarship (in brief, by doing \protect\hyperlink{open-science}{\textbf{Open Science}}). Our members endeavour to be careful and honest about their work, prioritizing the computational and empirical reproducibility of our research. Our research practices will strive to include:

\begin{itemize}
\tightlist
\item
  Completing pre-registrations and Registered Reports
\item
  Repeating and replicating experiments
\item
  Providing clean and commented code
\item
  Sharing data associated with our publications
\item
  Uploading conference posters and preprints for open access
\item
  Collaborating with local, national and overseas research groups
\end{itemize}

Further details and resources on these research practices are included in later chapters.

\hypertarget{our-research}{%
\section{Our research}\label{our-research}}

The specific goal of our research is to better understand attention and working memory - how we store information in mind for ongoing cognition. In particular, we examine how different perceptual and cognitive factors influence what is represented in mind and how that might be reflected in the brain. For example, in what ways does the complexity of a visual stimulus limit working memory? Do various forms of learning lead to changes in how information is encoded or maintained? Can we observe any changes in how working memory is operation in associated brain signals?

We collect experimental data using psychophysics and neuroimaging (electroencephalography; EEG) methods, and analyse the data in a variety of ways including computational modeling and machine learning. We will often develop and program novel behavioural paradigms using MATLAB (and sometimes Python), and conduct the statistical analysis using MATLAB and R (and sometimes Python). We can also collect data through online methods using jsPsych, Pavlovia and Proflic.

\hypertarget{lab-values}{%
\section{Our core values}\label{lab-values}}

The way we conduct ourselves is a critical part of how we conduct science. Lab members will strive to uphold the following values:

\begin{itemize}
\item
  \textbf{Integrity}: Like many things in the world, science is build on trust and honesty. We will pursue rigorous and transparent research, ensuring our work is credible and reliable.
\item
  \textbf{Humility}: By accepting that we may not decidely know or understanding everything, we are open to new perspectives. We are willing to admit we made mistakes and share what we have learned from them.
\item
  \textbf{Kindness}: Science involves skepticism and criticism, but that can be done with care and empathy. We hope to be respectful members of the scientific community, being considerate and supportive to our colleagues at all levels.
\item
  \textbf{Diversity}: Science should reflect the vast perspectives, experiences and backgrounds that exist in the world. Our lab hopes to be inclusive to all, and promote equity throughout science.
\end{itemize}

\hypertarget{location}{%
\section{Location}\label{location}}

The PALM Lab is located in the School of Psychology at the University of Adelaide. Our address is \textless insert here with Maps link.\textgreater{}

\hypertarget{lab-culture}{%
\chapter{Lab culture}\label{lab-culture}}

This chapter describes the structure and culture within the PALM Lab. The goal is to be clear about expectations for the members of the lab (ones that William has, as well as for each other) and importantly for William himself.

\begin{quote}
\emph{This section aims to provide a picture of what mentoring and working with me will be like in the PALM Lab. It will continually be in-progress, shaped by (and including) feedback from lab members and others alike. -- William}
\end{quote}

\hypertarget{lab-environment}{%
\section{Our lab environment}\label{lab-environment}}

The lab should be a welcoming space for all members to work and conduct research. Everyone should be respectful and kind to each other, providing support and encouragement at all times. We are a team that shares in our successes, rather than competes with each other to get ahead (\emph{and resists current incentive structures in science}). In fact, lab members are encouraged to take an interest in each others' work, learning from each other and collaborate on research.

Getting to know each other beyond being colleagues at work provides a strong basis for developing community within the lab. William encourages (and will help fund and/or organize) social events for the lab, but note that participation will always be optional. (Note that William will welcome any opportunity to take a quick break from work to grab a coffee.)

\begin{quote}
\emph{I am mindful that conflicts may arise between lab members or with others externally. I will do my best to \protect\hyperlink{conflicts}{mediate any issues} to the best resolution for all parties. To this end, I value clear communication - I will try my best to ensure the lab space is safe to bring up any disagreements or problems, especially so if they involve me. I hope that in a world that seems to be increasingly divided, we can nurture an environment that can resolve conflicts by listening to each other and having mutual resepct. -- William}
\end{quote}

It is the responsibility of all lab members to keep the lab space organized and tidy, and to maintain a good working environment for all. This includes:

\begin{itemize}
\tightlist
\item
  Cleaning any rubbish or waste found in common spaces
\item
  Putting lab equipment back in place after use
\item
  Being considerate with shared spaces or resources (including dividing shared spaces or resources equitably, and communicating if something occurs that may impact others)
\item
  Being mindful about taking up one's time or interrupting others' work in the lab (including the volume of nearby conversations)
\item
  Taking an interest in other lab members' research projects (such as paying attention during talks or lab meetings)
\item
  Offering to pilot each other's experiments or assist with data collection (such as gelling in EEG)
\end{itemize}

\hypertarget{lab-work}{%
\section{Expectations about work}\label{lab-work}}

It is important to maintain work-life balance (\emph{in the face of capitalistic academic structures}) to ensure one's health, enjoyment in doing science, and to prevent burnout. Do prioritize taking breaks during the day, managing life's responsibilities and keeping one's commitment to others.

Work hours are mostly flexible - the PALM Lab adopts a `hybrid model' where lab members are expected to spend at least a few days of the week working in lab. Ultimately, all lab members are responsible for managing and organizing their time to ensure research goals are met. You are not expected to work during evenings, weekends or public holidays. (\emph{The one possible exception may be during critical deadlines, such as for conferences or grants. William will be as mindful as he can be of this, and if he does ask for something, you can be upfront and say it is not possible.})

You are expected to stay home when you are sick, so that others (lab members, colleagues, participants) do not catch any illnesses. If this happens, please reschedule any meetings as soon as you can.

\begin{quote}
\emph{As your mentor, I will set regular check-in meetings with you to align expectations on what work needs to be achieved and if they are being met. These meetings will be an opportunity to communicate to me about any concerns or struggles, so that I am aware and can help support you or make plans to continue to make progress. -- William}
\end{quote}

\hypertarget{lab-mentorship}{%
\section{Mentorship}\label{lab-mentorship}}

Lab members are encouraged to seek mentorship from many people and places, such that they learn from a diverse set of perspectives and experience. This could be from a senior graduate student or postdoc in the PALM Lab or in another lab, from a research collaborator or co-supervisor, and of course, from William. All PALM Lab members are recommended to think about what \textbf{skills} they would like to learn, what \textbf{goals} they have in life and in science, and what may be an \textbf{inspiration} to them, and to seek out mentors who can help them with that (\emph{it may not always be William!})

\hypertarget{expectations-for-william-as-a-research-supervisor-will-mentor}{%
\subsection{Expectations for William as a research supervisor (\#will-mentor)}\label{expectations-for-william-as-a-research-supervisor-will-mentor}}

William is expected to provide mentorship that includes:

\begin{itemize}
\tightlist
\item
  Research training on experimental design, programming, data collection, data management, statistical analysis, computational modeling, manuscript writing, open scholarship, and further.
\item
  Providing feedback in a timely manner (such as comments on project ideas, approval on conference submissions, suggestions on talks, edits on manuscripts, etc.)
\item
  Giving thoughtful advice on and provide opportunities for achieving future success and preparing you for your the next step in your career (in science/academia or otherwise)
\item
  Creating a healthy learning environment that encourages asking questions without judgment
\item
  Being accessible both in-person and virtually (e-mail or otherwise), including regular meetings to discuss research (or anything else)
\end{itemize}

\begin{quote}
\_I hope to be a supportive supervisor for all PALM Lab members, fostering a working relationship that lasts throughout their journey in academia and science.

I will help all mentees navigate the obstacles and challenges of academia as best I can, and I hope to set an example of how to conduct impactful and rigorous science, through open scholarship practices and service for the community. -- William\_
\end{quote}

\hypertarget{expectations-for-phd-students}{%
\section{Expectations for PhD students}\label{expectations-for-phd-students}}

\hypertarget{expectations-for-honours-students}{%
\section{Expectations for Honours students}\label{expectations-for-honours-students}}

\hypertarget{conflicts}{%
\section{Resolving conflicts}\label{conflicts}}

Conflicts within the lab or with others can occur despite our best intentions and efforts to prevent them. The hope is that conflicts will be resolved through listening to each other, understanding each other's perspective and respectfully reaching an agreed upon course of action.

If there is any potential for disagreement or unhappiness that cannot be resolved by discussing with the other party, the first person to talk to is William. However, if William does not seem to be the right person (or the conflict is sourced by or with William), then the following contacts may be helpful:

\begin{itemize}
\tightlist
\item
  If you are a PhD student, the members of your thesis committee or the Dean of Learning
\item
  If you are not a PhD student, seek out \ldots.
\end{itemize}

\hypertarget{research-policies}{%
\chapter{Research Policies}\label{research-policies}}

\hypertarget{academic-integrity}{%
\section{Academic integrity}\label{academic-integrity}}

A core value of the PALM Lab and the University of Adelaide is to conduct scientific research with integrity. Lab members are expected to comply with the \href{https://www.adelaide.edu.au/policies/96/}{University of Adelaide's Responsible Conduct of Research Policy}. Research misconduct, such as fabrication (\emph{making up data or results}), falsification (\emph{changing or omitting methods or data such that the research record is inaccurate}), or plagiarism (\emph{taking another person's ideas or results without appropriate credit}) will not be tolerated. Lab members are encouraged to read these research policies and ask William if they have any questions.

Research fraud is entirely antithetical to the PALM Lab's commitment to benefit society by working towards the truth through rigorous and transparent research. There is no excuse for research fraud - it risks one's career, is a disservice to the field, and undermines society's trust in science.

\begin{quote}
\_Scientists are typically evaluated on the number of research publications, and the prestige of the journals they are published in. These incentives may compel someone to conduct research misconduct -- to speed up research and/or to produce a notable finding. Simply put, research misconduct and fraud is not right, and never worth it -- it will be discovered (\emph{if not by the PALM lab prior to publication}) and jeopardize any success one may have hoped to get, and it will cause substantial costs to the field and society.

I have a different definition of research success that I hope you will learn to share - in brief, I believe the best science will include thoughtfulness of the experimental design, transparency of the research process, availability of the code and the data, reliability of the findings, positive impact on society, and much more. I hope that you will pursue research with these values foremost. -- William\_
\end{quote}

Any concern of research misconduct within the PALM Lab should be reported to William immediately, who will act according to the \href{https://www.adelaide.edu.au/policies/96/?dsn=policy.document;field=data;id=7885;m=view}{University of Adelaide's Research Misconduct Procedure (2019)} on how complaints of research misconduct will be handled and investigated. Further resources can be found on the \href{https://www.adelaide.edu.au/staff/research/ethics-compliance-integrity/research-integrity}{University of Adelaide's page on Research Integrity}.

\hypertarget{contributorship}{%
\section{Contributorship, not authorship}\label{contributorship}}

In psychology and neuroscience, scientists are recognized through authorship on scientific publications. Being the \textbf{first author} of a publication signifies being the lead researcher and main contributor on the project, whereas the \textbf{last author} denotes the senior researcher on the project. In the remaining positions (second to second-last), order of authorship typically by (perceived) amount of contribution to the project.

Current standards for authorship (such as by the \href{https://www.icmje.org/recommendations/browse/roles-and-responsibilities/defining-the-role-of-authors-and-contributors.html}{International Committee of Medical Journal Editors} and the \href{https://www.apa.org/research/responsible/publication}{American Psychological Association}) require a writing contribution to the research publication. Instead, the PALM Lab believes that \textbf{all researchers contributing to a scientific publication should be recognized} (``a contributorship model''), and adopts the \href{https://credit.niso.org/}{Contributor Roles Taxonomy (CRediT)} to assign credit and authorship. CRediT lists 14 research roles that contribute to research outputs, listed below with brief explanations as how they might pertain to the PALM Lab:

\begin{itemize}
\tightlist
\item
  \textbf{Conceptualization} - The \textbf{formulation} or evolution of overarching research goals and aims.
\item
  \textbf{Data curation} - Data \textbf{management}, including generating metadata, cleaning and maintaining research data
\item
  \textbf{Formal analysis} - Application of statistical, mathematical, and/or computational techniques to \textbf{analyse experimental data}.
\item
  \textbf{Funding acquisition} - Acquisition of the financial support for conducting the research
\item
  \textbf{Investigation} - \textbf{Conducting the experiments} for data collection
\item
  \textbf{Methodology} - \textbf{Design} of the experiment
\item
  \textbf{Project administration} - \textbf{Coordination} of the research activity
\item
  \textbf{Resources} - \textbf{Provision} of any equipment
\item
  \textbf{Software} - \textbf{Programming} and testing of experimental code.
\item
  \textbf{Supervision} - \textbf{Oversight} for the research activity, including mentorship.
\item
  \textbf{Validation} - \textbf{Verification} of the overall reproducibility of results/experiments.
\item
  \textbf{Visualization} - \textbf{Preparation} of the published work, specifically data presentation.
\item
  \textbf{Writing -- original draft} - Preparation of the initial draft.
\item
  \textbf{Writing -- review and editing} - Critical review, commentary or revision.
\end{itemize}

The standard for authorship on any research publication from the PALM Lab is to have adequate involvement in \textbf{at least one} of the above roles, with the following as minimum requirements:

\begin{itemize}
\tightlist
\item
  \textbf{Investigation} - The researcher was responsible for data collection of at least a third (1/3) of the sample size in any experiment. (Assistance with EEG preparation does not count towards this.)
\item
  \textbf{Validation} - Conducted code review at least once, producing a report on the computational reproducibility of the results and experiment.
\item
  \textbf{Writing} - Provided comments or suggestions that both the first author and William agrees made a significant impact on the communication of the ideas or results. That is, the review went beyond grammatical edits or syntax changes.
\end{itemize}

If not explicitly required by the journal, the CRediT contributions will be listed in the acknowledgements or footnotes of all research outputs of the PALM Lab (see \protect\hyperlink{tenzing}{the tenzing app}. The first author is expected to \textbf{take ownership} of the research project and have contributed to the majority of these roles, likely \emph{conceptualization, data curation, formal analysis, investigation, project administration, software, visualization, writing -- original draft and writing -- review and editing}. In rare circumstances, there may be two or more authors that warrant lead authorship on a research publication, which may be recognized with co-first authorship. Order of authorship (other than first and senior) may be decided by counting the number of \emph{major} contributions made in each of the roles.

Authorship should be discussed as early as possible (such as at the beginning of a research project), so that expectations are aligned and contributions are equitably recognized for all. There are potential unforeseen circumstances that may lead to uncertainty about whether a researcher should be recognized with authorship - in these circumstances, the decision ultimately rests with the first author and William (who will lean towards inclusive authorship). In discussion with relevant parties, lead (first) authorship will be passed on if the responsibility for a research project (\emph{data, analyses and results}) is given to someone else (for example, if a lab member collected an experimental dataset but is unlikely to complete the analysis and publish the research).

\hypertarget{open-science}{%
\section{Open Science}\label{open-science}}

The PALM Lab aims to conduct rigorous and transparent research through Open Science practices. With William's guidance, each lab member is encouraged to pursue the following research practices as best as they can. Note that this is not an all-or-none expectation - Open Science is like a buffet, where scientists should strive to do what they can (Bergmann)

\hypertarget{preregistrations-and-registered-reports}{%
\subsection{Preregistrations (and Registered Reports)}\label{preregistrations-and-registered-reports}}

Preregistration is the practice of creating a public, timestamped record of one's experimental design and intended analyses \textbf{prior to} data collection. It is intended to promote the transparency of one's research and prevent questionable research practices like \emph{p}-hacking and \emph{HARK}ing. In a Registered Report format, the preregistration is peer-reviewed (known as Stage 1 of the Registered Report). Once accepted, the results will be published in-principle by the journal if the preregistration was followed.

Preregistration can be a useful practice because it necessitates clear rationales for decisions that are made in the research design process, curbing the potential to make idiosyncratic choices. This can help research students get on the same page as research supervisors (i.e.~William) on the aims of the research project, and to understand the reasons for the approach being taken.

PhD students and postdocs in the PALM Lab are encouraged to draft preregistrations with William (even if not intending to publish them) on the \href{https://osf.io/registries/osf/new}{Open Science Framework (OSF) Registries}. Honours students are not expected to draft preregistrations for the sake of timeliness of their empirical research project (though it may still be a useful exercises for writing the eventual thesis.)

\hypertarget{sharing-experiment-code-and-data}{%
\subsection{Sharing experiment code and data}\label{sharing-experiment-code-and-data}}

The PALM Lab aims to share all experimental code and data associated with research publications to promote reproducibility of the findings. In brief, open code and data allow peer reviewers to attempt empirical and/or computational reproductions of the projects, ensuring the rigor and credibility of the research, as well as enable secondary uses to increase the overall efficiency of science.

With William's guidance, all lab members are encouraged to write transparent code and manage data with the expectation that it will be publicly available when submitting the research manuscript for publication. (See \href{https://journals.plos.org/ploscompbiol/article?id=10.1371/journal.pcbi.1005510}{Wilson et al., 2017 for good enough practices for scientific computing}). The (de-identified) experiment data and code (both experiment and analysis) will be uploaded to the \href{https://osf.io}{Open Science Framework}. Neuroimaging (EEG) data may be uploaded to \href{https://openneuro.org/}{OpenNeuro} which requires the \href{https://bids.neuroimaging.io/}{Brain Imaging Data Structure (BIDS)} format for data storage. See \href{https://osf.io/wjr7u/}{this OSF repository} for an example.

\hypertarget{preprints-and-conference-presentations}{%
\subsection{Preprints (and conference presentations)}\label{preprints-and-conference-presentations}}

For the quick dissemination of research, the PALM Lab will upload preprints of research manuscripts. These will be uploaded to the \href{https://psyarxiv.com}{PsyArXiv} server, with links to any associated OSF repositories. The preprints (and any associated conference presentations) will also be made available on the \href{https://palm-lab.github.io/presentations}{PALM Lab website}. PALM Lab members will be recommended to upload the preprint when submitting their manuscript for publication at a scientific journal.

\hypertarget{lab-resources}{%
\chapter{Lab Resources}\label{lab-resources}}

This chapter provides useful links to software and associated resources for conducting research in the PALM Lab.

\hypertarget{required}{%
\section{Required}\label{required}}

The following are likely needed for conducting research in the PALM Lab.

\hypertarget{matlab}{%
\subsection*{MATLAB}\label{matlab}}
\addcontentsline{toc}{subsection}{MATLAB}

The code for conducting the experiments and data analysis will mostly be written in MATLAB. The University of Adelaide provides \href{https://www.mathworks.com/academia/tah-portal/university-of-adelaide-30536634.html}{access to download and use the licensed software}. If a lab member is an experienced programmer (i.e.~requiring little coding supervision from William), they can choose to write all experimental code in Python.

\hypertarget{psychtoolbox}{%
\subsubsection*{PsychToolbox}\label{psychtoolbox}}
\addcontentsline{toc}{subsubsection}{PsychToolbox}

The software we use for building psychology experiments (i.e.~displaying stimuli and collecting responses) comes from \href{http://psychtoolbox.org/}{PsychToolbox}. The Python alternative is PsychoPy.

\hypertarget{eeglab}{%
\subsubsection*{EEGLAB}\label{eeglab}}
\addcontentsline{toc}{subsubsection}{EEGLAB}

The software we use for analysing EEG data within the MATLAB environment is \href{https://sccn.ucsd.edu/eeglab/index.php}{EEGLAB}. Our artifact rejection pipeline can be found on Github.

\hypertarget{r-and-rstudio}{%
\subsection*{R and RStudio}\label{r-and-rstudio}}
\addcontentsline{toc}{subsection}{R and RStudio}

To run some statistical analyses, we use R and \href{https://posit.co/download/rstudio-desktop/}{RStudio}.

\href{https://r4ds.hadley.nz/}{R for Data Science by Hadley Wickham} is a free book that is a great introduction into using R for data analysis and visualization.

\hypertarget{github-and-github-desktop}{%
\subsection*{Github and Github Desktop}\label{github-and-github-desktop}}
\addcontentsline{toc}{subsection}{Github and Github Desktop}

For code management and version control, the lab uses Git and \href{https://github.com}{Github}. The \href{https://github.com/PALM-Lab}{PALM Lab's Github} will contain all code repositories for our lab, including shared packages for our common processing or analyses (such as artifact rejection of EEG).

Github's \href{https://docs.github.com/en/get-started/quickstart/hello-world}{own start-up guide} is a good place to start. For those less experienced with programming, \href{https://desktop.github.com/}{Github Desktop} is an easy-to-use interface for maintaining code between your local computer and Github.

\hypertarget{recommended}{%
\section{Recommended}\label{recommended}}

\hypertarget{zotero}{%
\subsection*{Zotero}\label{zotero}}
\addcontentsline{toc}{subsection}{Zotero}

\href{https://www.zotero.org/}{Zotero} is a software app for organizing scientific papers and managing references. It has an in-built reader with note taking capabilities.

\href{https://www.zotero.org/download/connectors}{Zotero Connector} is a very useful Google Chrome extension for adding journal articles directly into Zotero from Google Scholar or journal webpages.

\hypertarget{jamovi-or-jasp}{%
\subsection*{Jamovi or JASP}\label{jamovi-or-jasp}}
\addcontentsline{toc}{subsection}{Jamovi or JASP}

\hypertarget{tenzing}{%
\subsection*{Tenzing}\label{tenzing}}
\addcontentsline{toc}{subsection}{Tenzing}

\href{https://tenzing.club}{\emph{Tenzing}} is an easy-to-use web app for quickly generating the author list, their affiliations and their contributions according to the Contributor Roles Taxonomy (CRediT). The PALM Lab uses a \protect\hyperlink{contributorship}{policy of contributorship} to credit researchers. Use \emph{tenzing} to generate the CRediT roles prior to the submission of a research manuscript for publication.

The app was created by Alex Holcombe, Marton Kovacs, Frederik Aust and Balazs Aczel.

\href{https://journals.plos.org/plosone/article?id=10.1371/journal.pone.0244611}{Holcombe, A. O., Kovacs, M., Aust, F., \& Aczel, B. (2020). Documenting contributions to scholarly articles using CRediT and tenzing. PLoS ONE, 15(12), e0244611.}.

\hypertarget{optional}{%
\section{Optional}\label{optional}}

\hypertarget{obsidian}{%
\subsection*{Obsidian}\label{obsidian}}
\addcontentsline{toc}{subsection}{Obsidian}

\href{https://obsidian.md/}{Obsidian} is a note-taking and note management software that is designed to be a `second brain'. Notes are written and saved as Markdown files, and with the use of tags and links, can provide a powerful organization and navigation of the notes. Obsidian does have a steep learning curve and requires some setting up to work best for you.

\hypertarget{adobe-illustrator}{%
\subsection*{Adobe Illustrator}\label{adobe-illustrator}}
\addcontentsline{toc}{subsection}{Adobe Illustrator}

Adobe Illustrator is part of the Adobe Creative Cloud, and is useful for editing and cleaning up scientific figures for publication (for example, \emph{generating vector graphics to the appropriate sizes}), and for creating well-designed conference posters.

\hypertarget{miscellaneous}{%
\chapter{Miscellaneous}\label{miscellaneous}}

\hypertarget{templates}{%
\section{Templates}\label{templates}}

\hypertarget{sharing-your-book}{%
\chapter{Sharing your book}\label{sharing-your-book}}

\hypertarget{publishing}{%
\section{Publishing}\label{publishing}}

HTML books can be published online, see: \url{https://bookdown.org/yihui/bookdown/publishing.html}

\hypertarget{pages}{%
\section{404 pages}\label{pages}}

By default, users will be directed to a 404 page if they try to access a webpage that cannot be found. If you'd like to customize your 404 page instead of using the default, you may add either a \texttt{\_404.Rmd} or \texttt{\_404.md} file to your project root and use code and/or Markdown syntax.

\hypertarget{metadata-for-sharing}{%
\section{Metadata for sharing}\label{metadata-for-sharing}}

Bookdown HTML books will provide HTML metadata for social sharing on platforms like Twitter, Facebook, and LinkedIn, using information you provide in the \texttt{index.Rmd} YAML. To setup, set the \texttt{url} for your book and the path to your \texttt{cover-image} file. Your book's \texttt{title} and \texttt{description} are also used.

This \texttt{gitbook} uses the same social sharing data across all chapters in your book- all links shared will look the same.

Specify your book's source repository on GitHub using the \texttt{edit} key under the configuration options in the \texttt{\_output.yml} file, which allows users to suggest an edit by linking to a chapter's source file.

Read more about the features of this output format here:

\url{https://pkgs.rstudio.com/bookdown/reference/gitbook.html}

Or use:

\begin{Shaded}
\begin{Highlighting}[]
\NormalTok{?bookdown}\SpecialCharTok{::}\NormalTok{gitbook}
\end{Highlighting}
\end{Shaded}


  \bibliography{book.bib,packages.bib}

\end{document}
