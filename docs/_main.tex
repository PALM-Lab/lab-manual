% Options for packages loaded elsewhere
\PassOptionsToPackage{unicode}{hyperref}
\PassOptionsToPackage{hyphens}{url}
%
\documentclass[
]{book}
\usepackage{amsmath,amssymb}
\usepackage{iftex}
\ifPDFTeX
  \usepackage[T1]{fontenc}
  \usepackage[utf8]{inputenc}
  \usepackage{textcomp} % provide euro and other symbols
\else % if luatex or xetex
  \usepackage{unicode-math} % this also loads fontspec
  \defaultfontfeatures{Scale=MatchLowercase}
  \defaultfontfeatures[\rmfamily]{Ligatures=TeX,Scale=1}
\fi
\usepackage{lmodern}
\ifPDFTeX\else
  % xetex/luatex font selection
\fi
% Use upquote if available, for straight quotes in verbatim environments
\IfFileExists{upquote.sty}{\usepackage{upquote}}{}
\IfFileExists{microtype.sty}{% use microtype if available
  \usepackage[]{microtype}
  \UseMicrotypeSet[protrusion]{basicmath} % disable protrusion for tt fonts
}{}
\makeatletter
\@ifundefined{KOMAClassName}{% if non-KOMA class
  \IfFileExists{parskip.sty}{%
    \usepackage{parskip}
  }{% else
    \setlength{\parindent}{0pt}
    \setlength{\parskip}{6pt plus 2pt minus 1pt}}
}{% if KOMA class
  \KOMAoptions{parskip=half}}
\makeatother
\usepackage{xcolor}
\usepackage{longtable,booktabs,array}
\usepackage{calc} % for calculating minipage widths
% Correct order of tables after \paragraph or \subparagraph
\usepackage{etoolbox}
\makeatletter
\patchcmd\longtable{\par}{\if@noskipsec\mbox{}\fi\par}{}{}
\makeatother
% Allow footnotes in longtable head/foot
\IfFileExists{footnotehyper.sty}{\usepackage{footnotehyper}}{\usepackage{footnote}}
\makesavenoteenv{longtable}
\usepackage{graphicx}
\makeatletter
\def\maxwidth{\ifdim\Gin@nat@width>\linewidth\linewidth\else\Gin@nat@width\fi}
\def\maxheight{\ifdim\Gin@nat@height>\textheight\textheight\else\Gin@nat@height\fi}
\makeatother
% Scale images if necessary, so that they will not overflow the page
% margins by default, and it is still possible to overwrite the defaults
% using explicit options in \includegraphics[width, height, ...]{}
\setkeys{Gin}{width=\maxwidth,height=\maxheight,keepaspectratio}
% Set default figure placement to htbp
\makeatletter
\def\fps@figure{htbp}
\makeatother
\setlength{\emergencystretch}{3em} % prevent overfull lines
\providecommand{\tightlist}{%
  \setlength{\itemsep}{0pt}\setlength{\parskip}{0pt}}
\setcounter{secnumdepth}{5}
\usepackage{booktabs}
\ifLuaTeX
  \usepackage{selnolig}  % disable illegal ligatures
\fi
\usepackage[]{natbib}
\bibliographystyle{plainnat}
\IfFileExists{bookmark.sty}{\usepackage{bookmark}}{\usepackage{hyperref}}
\IfFileExists{xurl.sty}{\usepackage{xurl}}{} % add URL line breaks if available
\urlstyle{same}
\hypersetup{
  pdftitle={Lab Manual for the PALM Lab},
  pdfauthor={Dr William XQ Ngiam},
  hidelinks,
  pdfcreator={LaTeX via pandoc}}

\title{Lab Manual for the PALM Lab}
\author{Dr William XQ Ngiam}
\date{Last updated on 04-06-2024}

\begin{document}
\maketitle

{
\setcounter{tocdepth}{1}
\tableofcontents
}
\hypertarget{welcome}{%
\chapter*{Welcome!}\label{welcome}}
\addcontentsline{toc}{chapter}{Welcome!}

\includegraphics{images/banner.png}

This is the lab manual of the \textbf{P}erception, \textbf{A}ttention, \textbf{L}earning and \textbf{M}emory \textbf{Lab} at the University of Adelaide, run by Dr William Ngiam. The PALM Lab is a cognitive psychology and neuroscience lab researching how we keep information in mind for ongoing cognition. You can find out more about our research and team on \href{https://palm-lab.github.io}{our lab website}.

This lab manual is mainly written for prospective or current members of the PALM Lab, but is shared publicly as a potentially helpful resource for conducting research and/or writing a lab manual. Each chapter contains guidelines, policies, and useful resources for how the lab conducts our research. It is a continual work in progress, as the way we conduct our scientific research evolves. Please feel free to share your feedback by \href{mailto:palm.laboratory@gmail.com}{contacting us via email} and suggestion revisions by submitting an issue to \href{https://github.com/PALM-lab/lab-manual/issues}{this lab manual's Github repository}.

\begin{center}\rule{0.5\linewidth}{0.5pt}\end{center}

Thank you to the following who inspired this lab manual:

\begin{itemize}
\tightlist
\item
  \href{https://github.com/alylab/labmanual/blob/master/aly-lab-manual.pdf}{Aly Lab}
\item
  \href{https://www.cns.nyu.edu/malab/lablife.html}{Wei Ji Ma Lab}

  \begin{itemize}
  \tightlist
  \item
    \href{https://docs.google.com/document/d/1f6foaCkHiCkPKjO0gdKkkdm2dn1EKn5xRKZv-Ut5Dg4/edit?usp=sharing}{See also their statement on lab culture and expectations}
  \end{itemize}
\item
  \href{https://bookdown.org/jordandanielsnyder/LabManual/expectations-and-responsiblities.html}{Global Community Wellness Lab}
\end{itemize}

\begin{center}\rule{0.5\linewidth}{0.5pt}\end{center}

The PALM Lab Manual written by William Ngiam and members of the PALM Lab is licensed under CC BY-NC 4.0

\hypertarget{the-lab}{%
\chapter{The PALM Lab}\label{the-lab}}

This chapter gives an overview of the PALM Lab - the overarching goal of our lab, our approach to conducting science, and our values.

\hypertarget{lab-mission}{%
\section{Our mission}\label{lab-mission}}

Our attention - broadly, the ability to select which information to focus on and keep in mind - is critical for everyday cognition. But what we can attend to and keep in mind is limited, making what we do retain incredibly valuable. In today's digital age, an endless amount of information is available at our fingertips, with social media and artificial intelligence designing algorithms that disarm and capture our attention in various ways. The overarching goal of our lab is to conduct rigorous science that leads to a \textbf{better understanding of human attention and working memory}, with the hope that it will help improve the collective attention of society to tackle greater problems.

\hypertarget{lab-philosophy}{%
\section{Our philosophy of science}\label{lab-philosophy}}

We believe that science is best when it is conducted with integrity and with the common good in mind. The PALM Lab strives to uphold this belief through a commitment to rigorous and transparent scholarship (in brief, by doing \protect\hyperlink{open-science}{\textbf{Open Science}}). Our members endeavour to be careful and honest about their work, prioritizing the computational and empirical reproducibility of our research. Our research practices will strive to include:

\begin{itemize}
\tightlist
\item
  Completing pre-registrations and Registered Reports
\item
  Repeating and replicating experiments
\item
  Providing clean and commented code
\item
  Sharing data associated with our publications
\item
  Uploading conference posters and preprints for open access
\item
  Collaborating with local, national and overseas research groups
\end{itemize}

Further details and resources on these research practices are included in later chapters.

\hypertarget{our-research}{%
\section{Our research}\label{our-research}}

The specific goal of our research is to better understand attention and working memory - how we store information in mind for ongoing cognition. In particular, we examine how different perceptual and cognitive factors influence what is represented in mind and how that might be reflected in the brain. For example, in what ways does the complexity of a visual stimulus limit working memory? Do various forms of learning lead to changes in how information is encoded or maintained? Can we observe any changes in how working memory is operation in associated brain signals?

We collect experimental data using psychophysics and neuroimaging (electroencephalography; EEG) methods, and analyse the data in a variety of ways including computational modeling and machine learning. We will often develop and program novel behavioural paradigms using MATLAB (and sometimes Python), and conduct the statistical analysis using MATLAB and R (and sometimes Python). We can also collect data through online methods using jsPsych, Pavlovia and Proflic.

\hypertarget{lab-values}{%
\section{Our core values}\label{lab-values}}

The way we conduct ourselves is a critical part of how we conduct science. Lab members will strive to uphold the following values:

\begin{itemize}
\item
  \textbf{Integrity}: Like many things in the world, science is built on trust and honesty. We will pursue rigorous and transparent research, ensuring our work is credible and reliable.
\item
  \textbf{Humility}: By accepting that we may not decidedly know or understand everything, we are open to new perspectives. We are willing to admit we made mistakes and share what we have learned from them.
\item
  \textbf{Kindness}: Science involves skepticism and criticism, but that does not mean it cannot be done with care and empathy. We hope to be respectful members of the scientific community, being considerate and supportive to our colleagues at all levels.
\item
  \textbf{Diversity}: Science should reflect the vast perspectives, experiences and people that exist in the world. Our lab hopes to be inclusive to all, and promote equity throughout science.
\end{itemize}

\hypertarget{location}{%
\section{Location}\label{location}}

The PALM Lab is located in the Hughes Building on the North Terrace campus of the University of Adelaide.

\hypertarget{lab-culture}{%
\chapter{Lab culture}\label{lab-culture}}

This chapter describes the structure and culture within the PALM Lab. The goal is to be clear about expectations for the members of the lab (ones that I have, as well as for each other) and importantly for myself.

\begin{quote}
\emph{This section aims to provide a picture of what mentoring and working with me will be like in the PALM Lab. It will continually be in-progress, shaped by (and including) feedback from lab members and others alike. -- William}
\end{quote}

\hypertarget{lab-environment}{%
\section{Our lab environment}\label{lab-environment}}

The lab should be a welcoming space for all members to work and conduct research. Everyone should be respectful and kind to each other, providing support and encouragement at all times. We are a team that shares in our successes, rather than competes with each other to get ahead (\emph{and resists current incentive structures in science}). In fact, lab members are encouraged to take an interest in each others' work, learning from each other and collaborate on research.

Getting to know each other beyond being colleagues at work provides a strong basis for developing community within the lab. I encourage (and will help fund and/or organize) social events for the lab, but note that participation will always be optional. The goal is to have every member feel like they belong in the lab. (Note that I welcome any opportunity to take a quick break from work to grab a coffee and chat.)

\begin{quote}
\emph{I am mindful that conflicts may arise between lab members or with others externally. I will do my best to \protect\hyperlink{conflicts}{mediate any issues} to the best resolution for all parties. To this end, I value clear communication - I will try my best to ensure the lab space is safe to bring up any disagreements or problems, especially so if they involve me. I hope that in a world that seems to be increasingly divided, we can nurture an environment that can resolve conflicts by listening to each other and having mutual resepct. -- William}
\end{quote}

It is the responsibility of all lab members to keep the lab space organized and tidy, and to maintain a good working environment for all. This includes:

\begin{itemize}
\tightlist
\item
  Cleaning any rubbish or waste found in common spaces
\item
  Putting lab equipment back in place after use
\item
  Being considerate with shared spaces or resources (including dividing shared spaces or resources equitably, and communicating if something occurs that may impact others)
\item
  Being mindful about taking up one's time or interrupting others' work in the lab (including the \textbf{volume of nearby conversations})
\item
  Taking an interest in other lab members' research projects (such as paying attention during talks or lab meetings)
\item
  Offering to pilot each other's experiments or assist with data collection (such as gelling in EEG)
\end{itemize}

\hypertarget{lab-work}{%
\section{Expectations about work}\label{lab-work}}

It is important to maintain work-life balance (\emph{in the face of capitalistic academic structures}) to ensure one's health, enjoyment in doing science, and to prevent burnout. Do prioritize taking breaks during the day or when you need them, managing life's responsibilities and keeping one's commitment to others. Remember that doing your best will look different on each day, it does not mean always working at full speed.

Full-time research is essentially equivalent to a full-time job (i.e.~\textasciitilde40 hours/week, 5 days per week). The PALM Lab adopts a `hybrid model' where lab members are expected to spend at least three days of the week working on campus/in lab, and have the option to work either from home or in lab on the remaining days. The work hours themselves are somewhat flexible -- to ensure overlap between members in lab for collaboration and discussions, lab members are asked to have 10am to 3pm as part of their in-person hours. Ultimately, all lab members are responsible for managing and organizing their time to ensure research goals are met. You are not expected to work during evenings, weekends or public holidays. (\emph{The one possible exception may be during critical deadlines, such as for conferences or grants. William will be as mindful as he can be of this, and if he does ask for something, you can be upfront and say it is not possible.})

You are expected to stay home when you are sick, so that others (lab members, colleagues, participants) do not catch any illnesses. If this happens, please be mindful of others' time and reschedule any meetings as soon as you can.

\begin{quote}
\emph{I like to get out of the office and work at a nearby café - especially to code or write. It would be fun to co-work together as a lab! I find working with others keeps me accountable and productive. -- William}
\end{quote}

\hypertarget{requesting-time-off}{%
\subsection*{Requesting time off}\label{requesting-time-off}}
\addcontentsline{toc}{subsection}{Requesting time off}

Progress does not happen in a linear fashion -- working more or harder does not always lead to faster or better outputs. Breaks are necessary for preventing burnout, for developing fresh ideas and for consolidating learning. Relaxing is not a waste of time when it is an investment in your future self to recover and keep going!

I trust lab members to be responsible for maintaining work hours and ensuring research progress is being made, and won't be strictly policing your hours. However, please check-in with me about taking any time off (mainly so that I am aware, as well as to make plans for your time away as needed). I recognise that there is more to life than science and research, so as a supervisor, I hope to make sure that there is balance (both enough work is being done and enough life is being lived.)

\begin{quote}
\emph{As your mentor, I will set regular (\textasciitilde weekly) check-in meetings with you to align expectations on what work needs to be achieved and if they are being met. These meetings will be an opportunity to communicate to me about any concerns or struggles, so that I am aware and can help support you or make plans to continue to make progress. -- William}
\end{quote}

\hypertarget{expectations-around-communication}{%
\subsection{Expectations around communication}\label{expectations-around-communication}}

Communication should always be with kindness and respectful of each other's time. General guidelines for various forms of communication are as follows:

\begin{itemize}
\tightlist
\item
  For most communication about lab-related matters, the PALM Lab uses \textbf{Slack}.

  \begin{itemize}
  \tightlist
  \item
    Slack can cause a lot of interruption to workflows. For this reason, it is not expected for Slack to be open at all times; it is recommended curating (or turning off) notifications to help maintain their focus on work. However, Slack should be checked periodically (\textasciitilde twice a day) to ensure no messages are missed.
  \item
    The PALM Lab is on the free plan of Slack, which means \textbf{message history is restricted to 90 days}. As a result, communication in Slack should mostly be one-off in nature (quick questions/brief check-ins/notifications). Lab members are expected to maintain good organizational and note-taking practices for their projects.
  \item
    Lab members are encouraged to set statuses (e.g.~\emph{do not disturb}, \emph{in lab}, \emph{working from home}, \emph{on holiday}). I will also set statuses for a quick sign of his availability (e.g.~\emph{in office}, \emph{teaching}, etc.).
  \end{itemize}
\item
  For any important communication (e.g.~anything that might need documentation/record-keeping, such as where a decision is being made) or any messaging with a large group or external collaborators, use email.

  \begin{itemize}
  \tightlist
  \item
    Remember to be mindful of email etiquette - try to avoid spam by batching multiple questions into the one email, keeping them within the one thread.
  \end{itemize}
\item
  I will try to organise in-person meetings (\textasciitilde weekly) for discussing research projects (i.e.~ideation, results, etc.). I might drop by the lab sporadically to check-in with whoever is around (and will try to be mindful of placing pressure on being in lab.)
\item
  \textbf{For emergencies, do not hesitate to contact William by phone.}
\end{itemize}

\hypertarget{making-mistakes}{%
\subsection*{Making mistakes}\label{making-mistakes}}
\addcontentsline{toc}{subsection}{Making mistakes}

An important part of our lab culture (and science overall) is our attitude and response towards mistakes. In the PALM Lab, we recognize that mistakes (small or large) can happen despite our best efforts, and that it is unhelpful to respond poorly and place blame. We all make mistakes! In fact, mistakes should be treated as \textbf{moments that one can grow from}, and that others can learn from when we share them. I will make an effort to lead by example and share both present and past mistakes that I have made!

As a lab, we will prioritize careful, rigorous research rather than speedy, flashy research outputs, committing to open scholarship (see our \protect\hyperlink{open-science}{approach to Open Science in the next chapter}). This means that we take additional steps towards preventing mistakes (for example, piloting experiments and code review). Nevertheless, sometimes mistakes can (and will) pop up. The members of the PALM Lab will be encouraged to take ownership of their mistakes - not to hide them but share and work through them with others (especially when it impacts others, or influences the research outcomes).

\begin{quote}
\emph{Some mistakes can \emph{feel} very big - for example, a bug in the experiment means a large amount of time and resources have been wasted, or an incorrect analysis changes the results of the experiments dramatically - but these things do happen in science! I believe one should be challenging themselves and putting themselves in positions to make mistakes and fail, because that leads to growth. Any mistake that happens in the lab will partly be my own, so I hope you feel that you can share them with me when they do happen. What's most important is to correcting the mistakes in a transparent manner together. -- William}
\end{quote}

\hypertarget{lab-mentorship}{%
\section{Mentorship}\label{lab-mentorship}}

Lab members are encouraged to seek mentorship from many people and places, such that they learn from a diverse set of perspectives and experience. This could be from a senior graduate student or postdoc in the PALM Lab or in another lab, from a research collaborator or co-supervisor, and of course, from myself. All PALM Lab members are recommended to think about what \textbf{skills} they would like to learn, what \textbf{goals} they have in life and in science, and what may be an \textbf{inspiration} to them, and to seek out mentors who can help them with that (\emph{it may not always be me!}).

\hypertarget{will-mentor}{%
\subsection{Expectations for William as a research supervisor}\label{will-mentor}}

I am expected to provide mentorship that includes:

\begin{itemize}
\tightlist
\item
  Research training on experimental design, programming, data collection, data management, statistical analysis, computational modeling, manuscript writing, open scholarship, and further.
\item
  Providing feedback in a timely manner (such as comments on project ideas, approval on conference submissions, suggestions on talks, edits on manuscripts, etc.)
\item
  Giving thoughtful advice on and provide opportunities for achieving future success and preparing you for your the next step in your career (in science/academia or otherwise)
\item
  Creating a healthy learning environment that encourages you to attempt things yourself and to fail at, asking questions without judgment and gives you the space to develop
\item
  Being accessible both in-person and virtually (e-mail or otherwise), including regular meetings to discuss research (or anything else)
\end{itemize}

\begin{quote}
\emph{I hope to be a supportive supervisor for all PALM Lab members, fostering a working relationship that lasts throughout their journey in academia and science. I like to think of the mentor role as being a coach - to me, it gives the sense that the mentee has the agency in what they can achieve!}

\emph{I will help all mentees navigate the obstacles and challenges of academia as best I can, and I hope to set an example of how to conduct impactful and rigorous science, through open scholarship practices and service for the community. -- William}
\end{quote}

\hypertarget{mentorship-style}{%
\subsubsection*{Mentorship style}\label{mentorship-style}}
\addcontentsline{toc}{subsubsection}{Mentorship style}

I will always strive for my mentees to achieve success towards their goals. I try to balance being a hands-on supervisor (for example, getting involved with programming, data collection, data analysis, and further), while encouraging independence and self-directed learning. I operate on a so-called `open-door' model -- lab members are encouraged to reach out to me with any questions at any time, and to openly ask for the support they need. This includes, where appropriate, emotional support and encouragement while navigating challenges in science and academia.

I think I am fairly flexible in mentoring style, adapting to the mentee -- lab members are encouraged to `mentor up' \href{https://graduateschool.syr.edu/wp-content/uploads/2017/03/Lee-et-al..pdf}{(see this useful guide on mentoring up by Lee et al., p.141)}. This includes discussing and aligning expectations, asking for and accepting advice and feedback, and having a roadmap for personal development (which I could help you plan!). You can also read more on \protect\hyperlink{student-expectations}{expectations of students below}.

You can find out a little bit more about me on \href{https://williamngiam.github.io}{my personal academic website}.

\hypertarget{weaknesses}{%
\subsubsection*{Weaknesses}\label{weaknesses}}
\addcontentsline{toc}{subsubsection}{Weaknesses}

\begin{quote}
\emph{I believe one key characteristic of a good scientist is their humility, because it goes hand in hand with accepting one's imperfections and being open to change. I acknowledge that I have weakenesses and imperfect understsanding, which I list below so that you are better aware when working with me. -- William}
\end{quote}

\begin{itemize}
\tightlist
\item
  I set a high standard of science for myself and the work that I am involved in -- one that goes beyond the norm for the field. For example, I take the additional steps of code review, reproducibility checks, replication of experiments and preparation of data for open access, rather than seeing the end goal as a publication. This high standard can put pressure on my collaborators - for example, rethinking how things are being done and potentially slowing research outputs from reaching stages of progress.
\item
  I can see that this high standard might mean I appear overly critical, especially if I were to give critique without kindness or not to voice any positive feedback. My past mentees have not voiced this concern, but please do not hesitate to remind me to say what went well!
\item
  As I am newly starting the PALM Lab, I am inexperienced with conducting research specifically at the University of Adelaide, and am yet to secure my own external research funding.
\item
  I have a tendency to overcommit, having trouble saying `no' when new opportunities arise. This may mean I am spread thin at times, leading to less timely responses or feedback.
\item
  I value work-life balance, but do go through phases where I am doing `work' late or on weekends -- I am usually reading related papers, or getting fixated on fixing programming issues or solving other problems. This may lead to message or responses at odd times, which I hope does not lead to the expectation that you should also work odd hours. (I will be very clear about this!)
\end{itemize}

\hypertarget{student-expectations}{%
\section{Expectations for PhD students}\label{student-expectations}}

\emph{under construction}

\hypertarget{expectations-for-honours-students}{%
\section{Expectations for Honours students}\label{expectations-for-honours-students}}

\emph{under construction}

\hypertarget{conflicts}{%
\section{Resolving conflicts}\label{conflicts}}

Conflicts within the lab or with others can occur despite our best intentions and efforts to prevent them. The hope is that conflicts will be resolved through listening to each other, sharing and understanding all perspectives, and respectfully reaching an agreed upon course of action.

If there is any potential for disagreement or unhappiness that cannot be resolved by discussing with the other party, the first person to talk to is me. However, if I do not seem to be the right person (or the conflict is sourced by or with me), then the following resources and contacts may be helpful:

\begin{itemize}
\tightlist
\item
  Disclosing or reporting an incident of unacceptable behaviour to the \href{https://www.adelaide.edu.au/safer-campus-community/reporting-options}{Integrity Unit}.
\item
  The university's \href{https://www.adelaide.edu.au/counselling/}{Counselling Support} provides free, short-term help with issues affecting student life.
\item
  Contacting and seeking advice from the Research Coordinator (\href{https://researchers.adelaide.edu.au/profile/rachel.stephens}{Dr Rachel Stephens}), the Research Integrity Advisor (\href{https://researchers.adelaide.edu.au/profile/carolyn.semmler}{Professor Carolyn Semmler}), and/or the Postgraduate Coordinator (\href{https://www.adelaide.edu.au/directory/diana.dorstyn}{Associate Professor Diana Dorstyn}).
\end{itemize}

\hypertarget{student-wellbeing}{%
\section{Student wellbeing}\label{student-wellbeing}}

The University of Adelaide is highly committed to supporting its students and their wellbeing. The \href{https://www.adelaide.edu.au/student-hubs/hub-central}{Hub Central} located on Level 3 of the North Terrace Campus, underneath the Hughes Building, is a 24/7 place where students can seek advice with all things. The staff will be more than willing to direct you to where you can receive relevant support, if they can't provide it themselves.

\hypertarget{research-policies}{%
\chapter{Research Policies}\label{research-policies}}

\hypertarget{academic-integrity}{%
\section{Academic integrity}\label{academic-integrity}}

A core value of the PALM Lab and the University of Adelaide is to conduct scientific research with integrity. Lab members are expected to comply with the \href{https://www.adelaide.edu.au/policies/96/}{University of Adelaide's Responsible Conduct of Research Policy}. Research misconduct, such as fabrication (\emph{making up data or results}), falsification (\emph{changing or omitting methods or data such that the research record is inaccurate}), or plagiarism (\emph{taking another person's ideas or results without appropriate credit}) will not be tolerated. Lab members are encouraged to read these research policies and ask William if they have any questions.

Research fraud is entirely antithetical to the PALM Lab's commitment to benefit society by working towards the truth through rigorous and transparent research. There is no excuse for research fraud - it risks one's career, is a disservice to the field, and undermines society's trust in science.

\begin{quote}
\emph{Scientists are typically evaluated on the number of research publications, and the prestige of the journals they are published in. These incentives may compel someone to conduct research misconduct -- to speed up research and/or to produce a notable finding. Simply put, research misconduct and fraud is not right, and never worth it -- it will be discovered (\emph{if not by the PALM lab prior to publication}) and jeopardize any success one may have hoped to get, and it will cause substantial costs to the field and society.}

\emph{I have a different definition of research success that I hope you will learn to share - in brief, I believe the best science will include thoughtfulness of the experimental design, transparency of the research process, availability of the code and the data, reliability of the findings, positive impact on society, and much more. I hope that you will pursue research with these values foremost. -- William}
\end{quote}

Any concern of research misconduct within the PALM Lab should be reported to William immediately, who will act according to the \href{https://www.adelaide.edu.au/policies/96/?dsn=policy.document;field=data;id=7885;m=view}{University of Adelaide's Research Misconduct Procedure (2019)} on how complaints of research misconduct will be handled and investigated. Further resources can be found on the \href{https://www.adelaide.edu.au/staff/research/ethics-compliance-integrity/research-integrity}{University of Adelaide's page on Research Integrity}.

\hypertarget{contributorship}{%
\section{Contributorship, not authorship}\label{contributorship}}

In psychology and neuroscience, scientists are recognized through authorship on scientific publications. Being the \textbf{first author} of a publication signifies being the lead researcher and main contributor on the project, whereas the \textbf{last author} denotes the senior researcher on the project. In the remaining positions (second to second-last), order of authorship typically by (perceived) amount of contribution to the project.

Current standards for authorship (such as by the \href{https://www.icmje.org/recommendations/browse/roles-and-responsibilities/defining-the-role-of-authors-and-contributors.html}{International Committee of Medical Journal Editors} and the \href{https://www.apa.org/research/responsible/publication}{American Psychological Association}) require a writing contribution to the research publication. Instead, the PALM Lab believes that \textbf{all researchers contributing to a scientific publication should be recognized} (``a contributorship model''), and adopts the \href{https://credit.niso.org/}{Contributor Roles Taxonomy (CRediT)} to assign credit and authorship. CRediT lists 14 research roles that contribute to research outputs, listed below with brief explanations as how they might pertain to the PALM Lab:

\begin{itemize}
\tightlist
\item
  \textbf{Conceptualization} - The \textbf{formulation} or evolution of overarching research goals and aims.
\item
  \textbf{Data curation} - Data \textbf{management}, including generating metadata, cleaning and maintaining research data
\item
  \textbf{Formal analysis} - Application of statistical, mathematical, and/or computational techniques to \textbf{analyse experimental data}.
\item
  \textbf{Funding acquisition} - Acquisition of the financial support for conducting the research
\item
  \textbf{Investigation} - \textbf{Conducting the experiments} for data collection
\item
  \textbf{Methodology} - \textbf{Design} of the experiment
\item
  \textbf{Project administration} - \textbf{Coordination} of the research activity
\item
  \textbf{Resources} - \textbf{Provision} of any equipment
\item
  \textbf{Software} - \textbf{Programming} and testing of experimental code.
\item
  \textbf{Supervision} - \textbf{Oversight} for the research activity, including mentorship.
\item
  \textbf{Validation} - \textbf{Verification} of the overall reproducibility of results/experiments.
\item
  \textbf{Visualization} - \textbf{Preparation} of the published work, specifically data presentation.
\item
  \textbf{Writing -- original draft} - Preparation of the initial draft.
\item
  \textbf{Writing -- review and editing} - Critical review, commentary or revision.
\end{itemize}

The standard for authorship on any research publication from the PALM Lab is to have adequate involvement in \textbf{at least one} of the above roles, with the following as minimum requirements:

\begin{itemize}
\tightlist
\item
  \textbf{Investigation} - The researcher was responsible for data collection of at least a third (1/3) of the sample size in any experiment. (Assistance with EEG preparation does not count towards this.)
\item
  \textbf{Validation} - Conducted code review at least once, producing a report on the computational reproducibility of the results and experiment.
\item
  \textbf{Writing} - Provided comments or suggestions that both the first author and William agrees made a significant impact on the communication of the ideas or results. That is, the review went beyond grammatical edits or syntax changes.
\end{itemize}

If not explicitly required by the journal, the CRediT contributions will be listed in the acknowledgements or footnotes of all research outputs of the PALM Lab (see \protect\hyperlink{tenzing}{the tenzing app}. The first author is expected to \textbf{take ownership} of the research project and have contributed to the majority of these roles, likely \emph{conceptualization, data curation, formal analysis, investigation, project administration, software, visualization, writing -- original draft and writing -- review and editing}. In rare circumstances, there may be two or more authors that warrant lead authorship on a research publication, which may be recognized with co-first authorship. Order of authorship (other than first and senior) may be decided by counting the number of \emph{major} contributions made in each of the roles.

Authorship should be discussed as early as possible (such as at the beginning of a research project), so that expectations are aligned and contributions are equitably recognized for all. There are potential unforeseen circumstances that may lead to uncertainty about whether a researcher should be recognized with authorship - in these circumstances, the decision ultimately rests with the first author and William (who will lean towards inclusive authorship). In discussion with relevant parties, lead (first) authorship will be passed on if the responsibility for a research project (\emph{data, analyses and results}) is given to someone else (for example, if a lab member collected an experimental dataset but is unlikely to complete the analysis and publish the research). William will likely be the senior (final) author on the manuscript, if the research is primarily conducted within the PALM Lab.

You can view the University of Adelaide's Authorship Policy at \href{https://www.adelaide.edu.au/policies/3503}{this link}. Disputes over authorship can arise, and their resolution may require mediation from the Research Integrity Advisor (\href{https://researchers.adelaide.edu.au/profile/carolyn.semmler}{Professor Carolyn Semmler}), the Postgraduate Coordinator (\href{https://www.adelaide.edu.au/directory/diana.dorstyn}{Associate Professor Diana Dorstyn})

\hypertarget{open-science}{%
\section{Open Science}\label{open-science}}

The PALM Lab aims to conduct rigorous and transparent research through Open Science practices. With William's guidance, each lab member is encouraged to pursue the following research practices as best as they can. Note that this is not an all-or-none expectation - Open Science is like a buffet, where scientists should strive to do what they can (Bergmann)

\hypertarget{preregistrations-and-registered-reports}{%
\subsection{Preregistrations (and Registered Reports)}\label{preregistrations-and-registered-reports}}

Preregistration is the practice of creating a public, timestamped record of one's experimental design and intended analyses \textbf{prior to} data collection. It is intended to promote the transparency of one's research and prevent questionable research practices like \emph{p}-hacking and \emph{HARK}ing. In a Registered Report format, the preregistration is peer-reviewed (known as Stage 1 of the Registered Report). Once accepted, the results will be published in-principle by the journal if the preregistration was followed.

Preregistration can be a useful practice because it necessitates clear rationales for decisions that are made in the research design process, curbing the potential to make idiosyncratic choices. This can help research students get on the same page as research supervisors (i.e.~William) on the aims of the research project, and to understand the reasons for the approach being taken.

PhD students and postdocs in the PALM Lab are encouraged to draft preregistrations with William (even if not intending to publish them) on the \href{https://osf.io/registries/osf/new}{Open Science Framework (OSF) Registries}. Honours students are not expected to draft preregistrations for the sake of timeliness of their empirical research project (though it may still be a useful exercises for writing the eventual thesis.)

\hypertarget{sharing-experiment-code-and-data}{%
\subsection{Sharing experiment code and data}\label{sharing-experiment-code-and-data}}

The PALM Lab aims to share all experimental code and data associated with research publications to promote reproducibility of the findings. In brief, open code and data allow peer reviewers to attempt empirical and/or computational reproductions of the projects, ensuring the rigor and credibility of the research, as well as enable secondary uses to increase the overall efficiency of science.

With William's guidance, all lab members are encouraged to write transparent code and manage data with the expectation that it will be publicly available when submitting the research manuscript for publication. (See \href{https://journals.plos.org/ploscompbiol/article?id=10.1371/journal.pcbi.1005510}{Wilson et al., 2017 for good enough practices for scientific computing}). The (de-identified) experiment data and code (both experiment and analysis) will be uploaded to the \href{https://osf.io}{Open Science Framework}. Neuroimaging (EEG) data may be uploaded to \href{https://openneuro.org/}{OpenNeuro} which requires the \href{https://bids.neuroimaging.io/}{Brain Imaging Data Structure (BIDS)} format for data storage. See \href{https://osf.io/wjr7u/}{this OSF repository} for an example of a code and data repository.

The University of Adelaide provides support on achieving \href{https://www.adelaide.edu.au/library/library-services/services-for-researchers/fair}{FAIR (\emph{findable, accessible, interoperable and reusable}) data}, including this \href{https://www.adelaide.edu.au/technology/research/research-data/research-data-planner}{Research Data Planner} tool.

\hypertarget{preprints-and-conference-presentations}{%
\subsection{Preprints (and conference presentations)}\label{preprints-and-conference-presentations}}

For the quick dissemination of research, the PALM Lab will upload preprints of research manuscripts. These will be uploaded to the \href{https://psyarxiv.com}{PsyArXiv} server, with links to any associated OSF repositories. The preprints (and any associated conference presentations) will also be made available on the \href{https://palm-lab.github.io/presentations}{PALM Lab website}. PALM Lab members will be recommended to upload the preprint alongside submitting their manuscript for publication at a scientific journal.

\hypertarget{lab-resources}{%
\chapter{Lab Resources}\label{lab-resources}}

This chapter provides useful links to software and associated resources for conducting research in the PALM Lab. The following is not a strict list; lab members are encouraged to explore what tools and resources are out there, especially open source software, to keep current with research.

\hypertarget{required}{%
\section{Required}\label{required}}

The following resources will likely be needed for conducting research in the PALM Lab.

\hypertarget{matlab-or-python}{%
\subsection*{MATLAB (or Python)}\label{matlab-or-python}}
\addcontentsline{toc}{subsection}{MATLAB (or Python)}

The code for conducting the experiments and data analysis will mostly be written in MATLAB. The University of Adelaide provides \href{https://www.mathworks.com/academia/tah-portal/university-of-adelaide-30536634.html}{access to download and use the licensed software}.

If a lab member is an experienced programmer (i.e.~requiring \emph{not as much} coding supervision from William), they can opt to write their experimental and analysis code in \href{https://www.python.org/}{\textbf{Python}}, an open source software with numerous well-maintained packages for scientific research.

\hypertarget{psychtoolbox-or-psychopy}{%
\subsubsection*{PsychToolbox (or PsychoPy)}\label{psychtoolbox-or-psychopy}}
\addcontentsline{toc}{subsubsection}{PsychToolbox (or PsychoPy)}

\href{http://psychtoolbox.org/}{PsychToolbox} is the software package we use for building psychology experiments (i.e.~displaying stimuli and collecting responses) in MATLAB. A useful set of demos can be found on \href{https://peterscarfe.com/ptbtutorials.html}{Peter Scarfe's lab website}. The Python alternative is \href{https://www.psychopy.org/}{PsychoPy}.

\hypertarget{eeglab}{%
\subsubsection*{EEGLAB}\label{eeglab}}
\addcontentsline{toc}{subsubsection}{EEGLAB}

\href{https://sccn.ucsd.edu/eeglab/index.php}{EEGLAB} is the software we use for analysing electroencephalography (EEG) data within the MATLAB environment. EEGLAB provides a point-and-click interface to conduct most EEG analyses (such as calculating event-related potentials), as well as the base functions for custom analysis scripts. Our artifact rejection pipeline can be found on Github.

\href{https://mitpress.mit.edu/9780262525855/an-introduction-to-the-event-related-potential-technique/}{An introduction to the event-related potential by Steve Luck} is a good starting point for learning about EEG and various event-related potentials.

\hypertarget{r-and-rstudio}{%
\subsection*{R and RStudio}\label{r-and-rstudio}}
\addcontentsline{toc}{subsection}{R and RStudio}

To run some statistical analyses, we may use R and \href{https://posit.co/download/rstudio-desktop/}{RStudio}.

\href{https://r4ds.hadley.nz/}{R for Data Science by Hadley Wickham} is a free book that is a great introduction into using R for data analysis and visualization.

\hypertarget{github-and-github-desktop}{%
\subsection*{Github and Github Desktop}\label{github-and-github-desktop}}
\addcontentsline{toc}{subsection}{Github and Github Desktop}

For code management and version control, the lab uses Git and \href{https://github.com}{Github}. The \href{https://github.com/PALM-Lab}{PALM Lab's Github} will contain all code repositories for our lab, including shared packages for our common processing or analyses (such as artifact rejection of EEG).

Github's \href{https://docs.github.com/en/get-started/quickstart/hello-world}{own start-up guide} is a good place to learn how to use Github. For those less experienced with programming, \href{https://desktop.github.com/}{Github Desktop} is an easy-to-use interface for maintaining code between your local computer and Github.

\hypertarget{recommended}{%
\section{Recommended}\label{recommended}}

\hypertarget{zotero}{%
\subsection*{Zotero}\label{zotero}}
\addcontentsline{toc}{subsection}{Zotero}

\href{https://www.zotero.org/}{Zotero} is a desktop software app for organizing scientific papers and managing references. It has an in-built PDF reader with tools for annotations and note-taking, and plug-ins for inserting references and bibliographies into manuscripts.

\href{https://www.zotero.org/download/connectors}{Zotero Connector} is a very useful Google Chrome extension for adding journal articles directly into Zotero from Google Scholar or journal webpages. This plug-in will also connect Zotero to Google Docs, enabling adding citations and bibliographies into any working manuscript (which can be ported to Microsoft Word at any time). An \href{https://www.zotero.org/support/word_processor_plugin_usage}{additional extension} is needed to connect Zotero to Microsoft Word.

\hypertarget{jamovi-or-jasp}{%
\subsection*{Jamovi or JASP}\label{jamovi-or-jasp}}
\addcontentsline{toc}{subsection}{Jamovi or JASP}

\href{https://jasp-stats.org/}{JASP} and \href{https://www.jamovi.org/}{Jamovi} are free, open-source statistical software programs that provide a point-and-click interface for conducting both frequentist and Bayesian statistical analyses.

\hypertarget{tenzing}{%
\subsection*{Tenzing}\label{tenzing}}
\addcontentsline{toc}{subsection}{Tenzing}

\href{https://tenzing.club}{\emph{Tenzing}} is an easy-to-use web app for quickly generating the author list, their affiliations and their contributions according to the Contributor Roles Taxonomy (CRediT). The PALM Lab uses a \protect\hyperlink{contributorship}{policy of contributorship} to credit researchers. Use \emph{tenzing} to generate the CRediT roles prior to the submission of a research manuscript for publication.

The app was created by Alex Holcombe, Marton Kovacs, Frederik Aust and Balazs Aczel. See \href{https://journals.plos.org/plosone/article?id=10.1371/journal.pone.0244611}{Holcombe, A. O., Kovacs, M., Aust, F., \& Aczel, B. (2020). Documenting contributions to scholarly articles using CRediT and tenzing. PLoS ONE, 15(12), e0244611.}.

\hypertarget{optional}{%
\section{Optional}\label{optional}}

\hypertarget{obsidian}{%
\subsection*{Obsidian}\label{obsidian}}
\addcontentsline{toc}{subsection}{Obsidian}

\href{https://obsidian.md/}{Obsidian} is a note-taking and knowledge management software that is designed to be a `second brain' for the user. Notes are written and saved as Markdown files, and with the use of tags, links and community-built plugins, Obsidian can provide a powerful organization and navigation of the notes. Obsidian does have a steep learning curve and requires some setting up (including thought about structure and organization) before it can work well.

\hypertarget{adobe-illustrator}{%
\subsection*{Adobe Illustrator}\label{adobe-illustrator}}
\addcontentsline{toc}{subsection}{Adobe Illustrator}

Adobe Illustrator is part of the Adobe Creative Cloud, and is useful for editing and cleaning up scientific figures for publication (for example, \emph{generating vector graphics to the appropriate sizes}), and for creating well-designed conference posters.

Additional resources (mostly educational in nature) can be found in \protect\hyperlink{extra-resources}{Chapter 6}.

\hypertarget{getting-started}{%
\chapter{Getting started}\label{getting-started}}

This chapter is to outline first steps to take for anyone joining the PALM Lab (\emph{welcome!}).

\hypertarget{your-first-week}{%
\section{Your first week}\label{your-first-week}}

\begin{enumerate}
\def\labelenumi{\arabic{enumi}.}
\tightlist
\item
  If you're not already a University of Adelaide student, you will need a student ID\ldots{}
\item
  Join our Slack
\item
  Join our Google Calendars
\item
  Create a Github account (if you don't have one already) and join our \href{https://github.com/PALM-Lab}{lab Github}
\item
  Get added to the \href{https://palm-lab.github.io}{lab website}. If you are confident with the code, feel free to add your Markdown page to the website repo. If not, ask William to help add you.
\item
  Get ethics approval to conduct research. You will need to complete compulsory training and be added on to one of the lab's existing Institutional Review Board (IRB) research protocols (or create a new IRB protocol if a new research line is being started).
\end{enumerate}

\emph{under construction}

\hypertarget{miscellaneous}{%
\chapter{Miscellaneous}\label{miscellaneous}}

\hypertarget{collaborators}{%
\section{Collaborators}\label{collaborators}}

\hypertarget{the-university-of-adelaide}{%
\subsection{The University of Adelaide}\label{the-university-of-adelaide}}

The University of Adelaide has a collaborative research culture, with many working on understanding brain and behaviour.

\hypertarget{awhvogel-lab-at-the-university-of-chicago}{%
\subsection{Awh/Vogel Lab at the University of Chicago}\label{awhvogel-lab-at-the-university-of-chicago}}

William was a postdoctoral researcher with Prof Edward Awh and Prof Edward Vogel at the University of Chicago, and has developed a long-time working relationship with them. Expect a lot of collaboration and discussion with them.

\hypertarget{holclab-at-the-university-of-sydney}{%
\subsection{holcLab at the University of Sydney}\label{holclab-at-the-university-of-sydney}}

For his PhD, William was supervised by Prof Alex Holcombe at the University of Sydney, who gave his start in visual perception research and inspired his advocacy for open scholarship. The holcLab is interested in the limits of visual attention and uses psychophysical paradigms like the multiple-object tracking (MOT).

\hypertarget{extra-resources}{%
\section{Additional resources}\label{extra-resources}}

This following is a curated list of potentially useful resources that are relevant to the PALM Lab's ongoing research, or cognitive psychology and neuroscience more broadly.

\hypertarget{generalized-linear-models-linear-mixed-effects-models}{%
\subsection{Generalized linear models / linear mixed-effects models}\label{generalized-linear-models-linear-mixed-effects-models}}

\begin{itemize}
\tightlist
\item
  \href{https://journals.sagepub.com/doi/full/10.1177/2515245920960351}{Brown, V. A. (2021). An introduction to linear mixed-effects modeling in R. Advances in Methods and Practices in Psychological Science, 4(1), 2515245920960351.}
\item
  \href{http://mfviz.com/hierarchical-models/}{A visual introduction to hierarchical modeling} (also known as linear mixed effects) by Michael Freeman.
\end{itemize}

\hypertarget{open-scholarship}{%
\subsection{Open scholarship}\label{open-scholarship}}

\begin{itemize}
\tightlist
\item
  \href{https://the-turing-way.netlify.app/index.html}{The Turing Way} - a community-driven handbook to reproducible, ethical and collaborative data science. (\textbf{\emph{Highly recommended}})
\item
  \href{https://williamngiam.github.io/reading_lists/}{Reading lists on various Open Science topics created by William for ReproducibiliTea}
\end{itemize}

\hypertarget{lab-templates}{%
\subsection{Lab templates}\label{lab-templates}}

\hypertarget{some-favorite-inspirations}{%
\subsection{Some favorite inspirations}\label{some-favorite-inspirations}}

\begin{itemize}
\tightlist
\item
  \href{https://youtu.be/CT2vwm4oY00}{A scene about growth from The Bear, a fantastic TV series based in my second home, Chicago.}
\end{itemize}

  \bibliography{book.bib,packages.bib}

\end{document}
